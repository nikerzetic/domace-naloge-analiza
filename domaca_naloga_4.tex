\documentclass[12pt,a4paper,slovene]{article}
\usepackage [slovene]{babel}
\usepackage[utf8]{inputenc}

\usepackage{enumerate}
\usepackage{amsmath}
\usepackage{mathtools}
\usepackage{amssymb}

\newcommand{\NN}{\mathbb{N}}
\newcommand{\RR}{\mathbb{R}}
\newcommand{\ZZ}{\mathbb{Z}}

\begin{document}
\begin{enumerate}[(1)]
	\item Poenostavi naslednje izraze
	\begin{enumerate}[(a)]
		\item $(3+4i)\overline{(1-3i)}$
		\begin{multline*}
			(3+4i)\overline{(1-3i)} = (3+4i)(1+3i) = 3 + 9i + 4i - 12 = -9 + 13i
		\end{multline*}
		
		\item $\dfrac{7 - 3i}{1 + i}$
		
		\begin{multline*}
			\dfrac{7-3i}{1+i} = \dfrac{(7-3i)(1-i)}{(1+i)(1-i)} = \dfrac{7 - 7i - 3i - 3}{2} = \dfrac{4 - 10i}{2} = 2 - 5i
		\end{multline*}
		
		\item $(1 + i \sqrt{3})^{10}$
		
		To ena"cbo najla"zje re"simo tako, da jo pretvorimo v eulerjev zapis.
		\begin{align*}
		z & = 1 + i \sqrt{3}\\
		|z| &= \sqrt{1 + 3} = 2 \\
		\tan \varphi &= \dfrac{\sqrt{3}}{1} \Rightarrow \varphi = \dfrac{\pi}{3}\\
		z &= 2e^{i\frac{\pi}{3}}
		\end{align*}
		Nato pora"cunamo njeno deseto potenco:
		\begin{equation*}
		z^{10} = 2^{10}e^{10i\frac{\pi}{3}} = 1024e^{i\frac{4\pi}{3}}
		\end{equation*}
		Nato lahko to pretvorimo nazaj v kartezi"cno obliko:
		\begin{equation*}
		z^{10} = 1024\left(\cos\dfrac{4\pi}{3} + i\sin\dfrac{4\pi}{3}\right) =  1024\left(-\dfrac{1}{2} - i\dfrac{\sqrt{3}}{2}\right) = -512\left(1 + i\sqrt{3}\right)
		\end{equation*}
	\end{enumerate}

	\item Zapi"si vse vrednosti izraza $\sqrt[7]{(-\sqrt{3} - i)^5}$ in jih nari"si.
	
	To nalogo ponovno najla"zje re"simo tako, da pretvorimo v eulerjev zapis.
	\begin{align*}
	w &= -\sqrt{3} - i\\
	z^7 &= (-\sqrt{3} - i)^5 = w^5
	\end{align*}
	Re"sujemo za $z$.
	
	Najprej pora"cunamo $w^5$:
	\begin{align*}
	w &= -\sqrt{3} - i\\
	|w| &= \sqrt{3 + 1} = 2\\
	\tan \varphi &= \dfrac{-1}{-\sqrt{3}} \Rightarrow \varphi = \dfrac{\pi}{6}\\
    \end{align*}
    \textbf{Pozor!} Tule se skriva pogosta napaka. Nismo izračunali pravega kota. Če bi narisali skico, bi takoj opazili, da pri številu $w = -\sqrt{3} - i$ kot zagotovo ne more biti $\frac{\pi}{6}$! Pravi kot je večji za $\pi$, torej je $\varphi = \frac{7 \pi}{6}$. Sedaj lahko zapišemo $w$ v eulerjevi obliki.
    \begin{align*}
    w &= 2e^{i\frac{7\pi}{6}}\\
    w^5 &= 2^5 e^{i\frac{35\pi}{6}} = 32 e^{i\frac{35\pi}{6}}\\
    \intertext{najti želimo 7-mi koren tega števila}
    z &= |z|e^{i\varphi}\\
	z^7 &= w^5\\
	|z|^7 e^{i7\varphi} &= 32 e^{i(\frac{35\pi}{6} + 2k\pi)}\qquad k \in \ZZ
	\end{align*}
	Sedaj primerjamo absolutne vrednosti in kote posebej. Za absolutne vrednosti dobimo:
	\begin{equation*}
	|z|^7 = 32 \Rightarrow |z| = \sqrt[7]{32}
	\end{equation*}
	Za kote pa dobimo:
	\begin{align*}
	7\varphi &= \frac{35\pi}{6} + 2k\pi\\
	\varphi &= \frac{5\pi}{6} + \dfrac{2k\pi}{7},\ k \in \{0, 1, 2, \ldots , 6\}
	\end{align*}
	$k$ je celo število od $0$ do $6$, ker ko je $k = 7$ pri"stejemo kotu $2\pi$ in iz trigonometrije vemo, da smo nazaj na za"cetku.
	
	$z$ lahko sedaj zapi"semo kot:
	\begin{equation*}
	z = \sqrt[7]{32} e^{i\left(\frac{5\pi}{6} + \frac{2k\pi}{7}\right)} = \sqrt[7]{32}\left(\cos \left(\dfrac{5\pi}{6} + \dfrac{2k\pi}{7}\right) +i\sin \left(\dfrac{5\pi}{6} + \dfrac{2k\pi}{7} \right)\right)
	\end{equation*}
	 	
	\item Skiciraj naslendje podmno"zice kompleksnih "stevil.
	
    \begin{enumerate}[(a)]
        \item $\{z \in \mathbb{C}; 2 (\text{Re } z)^2 + \text{Im } < 1 \}$\\
        Zapišimo $z$ v kartezični obliki, $z = x + iy$. Vidimo, da so v množici tista kompleksna števila, ki ustrezajo neenačbi:
        \[
        2x^2 + y < 1
        \]
        Enačbo še malo preuredimo, in že znamo narisati to množico. Tu je žal ne bom narisal, ker trenutno nimam preveč časa.
        \[
        y < -2x^2 + 1
        \]
        Rezultat so vse točke, ki ležijo pod parabolo $y = -2x^2 + 1$.
        
        \item $\{ z \in \mathbb{C}; \text{Re }z^2 + 4 \text{Im }z = 0 \}$
        Naredimo enako kot pri prejšnji nalogi, samo da imamo celo nekaj dela z računanjem.
        \begin{align*}
            \text{Re}(x + iy)^2 + 4y &= 0\\
            \text{Re}(x^2 - y^2 + 2xyi) + 4x = x^2 - y^2 + 4y &= 0\\
            x^2 - (y - 2)^2 + 4 &= 0\\
            x^2 - (y - 2)^2 &= -4\\
            \frac{x^2}{4} - \frac{(y-2)^2}{4} &= -1
        \end{align*}
        Po tem, ko dopolnimo popolni kvadrat je očitno, da gledamo v enačbo hiperbole. Risanje je prepuščeno bralcu.
    \end{enumerate}
	
	\item Naj bo $k \neq 0$ realno "stevilo. Za katere $k$ je $1 + ik$ bli"ze izhodi"s"cu kot $1-\frac{i}{k}$?
	
	Razdaljo od izhodi"s"ca nam predstavlja absolutna vrednost "stevila, torej lahko zapi"semo:
	\begin{equation*}
	|1+ik| < \left|1-\dfrac{i}{k}\right|
	\end{equation*}
	Ker so absolutne vrednosti ve"cje ali enake 0, lahko neena"cbo kvadriramo in razre"simo absolutne vrednost:
	\begin{equation*}
	1 + k^2 < 1 + \dfrac{1}{k^2}
	\end{equation*}
	Sedaj lahko ena"cbo nekoliko preuredimo in dobimo:
	\begin{align*}
	k^2 &< \dfrac{1}{k^2}\\
	k^4 &< 1
	\end{align*}
	Ker je $k^2$ pozitivno "stevilo, lahko ena"cbo korenimo in dobimo:
	\begin{equation*}
	k^2 < 1
	\end{equation*}
	Sledi dolg premislek o tem, kaj je na"s rezultat. "Ce si skiciramo parabolo $k^2 = 1$, dobimo dve ni"cli: $-1, 1$, parabola pa je med tema dvema ni"clama manj"sa od 1. Torej je $k \in (-1, 1)$, ali druga"ce povedano: $|k| < 1$.
    
    Seveda tudi pri tej nalogi obstaja lahka očitna geometrijska rešitev. Narišemo si premico, na kateri ležijo vsa kompleksna števila, $1 + ik$, ko $k$ preteče vsa realna števila (razen 0). To je premica $x = 1$. Hitro opazimo da, manjši kot je $|k|$, manjša je oddaljenost od izhodišča. Sedaj nas zanima samo še za katere $k$ velja:
    \[
    |k| < \frac{1}{|k|}
    \]
    Očitno to velja samo za števila $|k|$, ki so $\leq 1$.

    \item Pokaži, da je množica kompleksnih števil $\left\{ z = \frac{3}{2 + \cos \varphi + i \sin \varphi}; \varphi \in \RR \right\}$ podmnožica krožnice s središčem $s = 2$ in polmerom 1.
    
    Kako se lotiti te naloge? Ali lahko preoblikujemo zgornjo enačbo tako, da bomo dobili del krožnice? Nevem. Lotimo se drugače. Najprej zapišimo, kaj velja za krožnico, omenjeno v besedilu:
    \[
    |z - 2| = 1
    \]
    Sedaj pa vstavimo omenjen izraz. Če je dana množica podmnožica krožnice, potem bo enakost veljala za vsak $\varphi \in \RR$.
    \[
    \left|\frac{3}{2 + \cos \varphi + i \sin \varphi} - 2\right| = \left|\frac{3}{2 + e^{i \varphi}} - 2\right| = \left|\frac{3 - 4 - 2 e ^{i\varphi}}{2 + e^{i \varphi}} \right| = \left|\frac{-1 - 2 e ^{i\varphi}}{2 + e^{i \varphi}} \right|
    \]
    No, kaj pa zdaj? Kako lahko izraz poenostavimo še bolj? Poskusimo uporabiti tole formulo za absolutno vrednost kompleksnega števila.
    \[
    |z| = \sqrt{z \overline{z}}
    \]
    Seveda vemo tudi, da je absolutna vrednost ulomka enaka kvocientu absolutnih vrednosti števca in imenovalca.
    \[
    \frac{|-1 - 2e^{i\varphi}|}{|2 + e^{i\varphi}|} \\
    \]
    Poračunajmo števec in imenovalec posebej. Spomnimo, da velja $\overline{e^{i\varphi}} = e^{-i\varphi}$.
    \begin{multline*}
    \sqrt{(-1 - 2e^{i\varphi}) \overline{(-1 - 2e^{i\varphi})}} =
    \sqrt{(1 + 2e^{i\varphi}) (1 + 2e^{-i\varphi})} = \\
    \sqrt{1 + 2e^{-i\varphi} + 2e^{i\varphi} + 4} = \sqrt{5 + 2(e^{-i\varphi} + e^{i\varphi})}
    \end{multline*}
    \begin{multline*}
    \sqrt{(2 + e^{i\varphi}) (2 - e^{-i\varphi})} =
    \sqrt{1 + 2e^{-i\varphi} + 2e^{i\varphi} + 4} = \sqrt{5 + 2(e^{-i\varphi} + e^{i\varphi})}
    \end{multline*}
    Vidimo, da sta števec in imenovalec enaka, torej, torej velja:
    \[
    \frac{\sqrt{5 + 2(e^{-i\varphi} + e^{i\varphi})}}{\sqrt{5 + 2(e^{-i\varphi} + e^{i\varphi})}} = \frac{1}{1} = 1
    \]
    
    
    
    \item Izračunaj množice kompleksnih števil $z$, ki zadoščajo (ne)enačbam. Množice rešitev nariši.
    
    \begin{enumerate}[(a)]
        \item $z^3 = -2 + 2i$
        Obe strani enačbe zapišemo z eulerjevo formulo in primerjamo absolutno vrednost ter kot.
        \begin{align*}
            |-2 + 2i| &= \sqrt{(-2)^2 + 2^2} = 2\sqrt{2}\\
            \tan \theta &= \left(\frac{y}{x}\right) = -\frac{2}{2} = -1 \implies \theta = \frac{3\pi}{4}\\
            \left(|z| e^{i \varphi}\right) &= |z|^3 e^{i 3\varphi} = 2\sqrt{2} e^{i\frac{3\pi}{4}}\\
            |z|^3 &= 2\sqrt{2} = \sqrt{8} \implies |z| = \sqrt[6]{8} = \sqrt{2}\\
            3\varphi &= \frac{3\pi}{4} + 2k\pi, k \in \{0, 1, 2\}\\
            \varphi &= \frac{\pi}{4} + \frac{2 k \pi}{3}              
        \end{align*}
        Sedaj lahko še napišemo števila z, ki ustrezajo enačbi, vendar v se ni treba preveč mučit, napišimo samo tako, kot je v rešitvah:
        \[
        z = \sqrt{2}(\cos(\frac{\pi}{4} + \frac{2k\pi}{3}) + i \sin(\frac{\pi}{4} + \frac{2k\pi}{3})),\ k = 0, 1, 2
        \]
        
        \item $z^8 + z^4 - 12 = 0$\\
        Poskusimo podobno, vendar preden se vržemo v nalogo lahko naredimo preprosto substitucijo, da nam bo lažje. Naj bo $w = z^4$. Tako moramo rešiti samo preprosto kvadratno enačbo.
        \begin{align*}
            w^2 + w - 12 &= 0\\
            (w + 4) (w - 3) &= 0 \implies w = -4 \lor w = 3 
        \end{align*}
        Zdaj samo poračunamo za oba primera.
        \begin{enumerate}
            \item $w=-4$
            \begin{align*}
                z^4 &= -4\\
                |z| &= \sqrt[4]{4} = \sqrt{2}\\
                \varphi &= \frac{\pi}{4} + \frac{2\pi}{4} = \frac{pi}{4} + \frac{\pi}{2}\\
                z &= \sqrt{2} (\cos(\frac{\pi}{4} + \frac{\pi}{2}) + i \sin (\frac{\pi}{4} + \frac{\pi}{2})), \ k = 0, 1, 2, 3
            \end{align*}
            \emph{Opomba:} Če bi slučajno pomislil, potem bi opazil, da so tole precej lepi koti, prav tako razdalja od izhodišča, vse nam je že od nekod znano. Lahko pišemo tudi:
            \[
            z = \pm 1 \pm i
            \]
            \item $w = 3$\\
            Identično kot za prvi primer, samo da se meni ne da toliko pisat.
            \begin{align*}
                z &= \sqrt{3} (\ldots)\\
                \varphi &= \frac{\pi}{4} + \frac{\pi}{2}\\
                z &= \sqrt{3} (\cos\varphi + i\sin \varphi),\ k = 1, 2, 3, 4\\
            \end{align*}
        \end{enumerate}
        
        \item $\sqrt{|z|^2 - 2} + \text{Im}\;z > \text{Re}\;z$\\
        Poskusimo zapisati $z$ v kartezični obliki, torej $z = x + yi$:
        \begin{align*}
            \sqrt{x^2 + y^2 - 2} + y > x\\
            \sqrt{x^2 + y^2 - 2} > x - y\\
        \end{align*}
        Poglejmo, kdaj ima ta enačba sploh smisel. Pogledati moramo, kdaj je vrednost znotraj korena negativna, in kakšna je desna stran.
        %
        \begin{enumerate}
            \item $x - y < 0$: Desna stran enačbe je negativna, leva pa bo vedno nenegativna, torej bo neenačaj veljal na celotnem definicijskem območju, torej ko velja $x^2 + y^2 - 2 \geq 0$.
            \[
            (x < y) \land (x^2 + y^2 \geq 2)
            \]
            \item $x - y \geq 0$: Enačbo lahko kvadriramo.
            \begin{align*}
                x^2 + y^2 - 2 &> (x - y)^2 = x^2 - 2xy +  y^2\\
                -2 &> -2xy\\
                xy &> 1\\
                (x \geq y) &\land (xy > 1)
            \end{align*}
        \end{enumerate}
        Torej imamo celotno rešitev:
        \[
        \left\{ z = x+ iy; (x \geq y) \land (xy > 1) \lor (x < y) \land (x^2 + y^2 \geq 2) \right\}
        \]
    \end{enumerate}

    \item Pokaži, da sta množici kompleksnih števil enaki:
    \[
    A = \left\{ z \in \mathbb{C}; \frac{1}{z} + \frac{1}{\overline{z}} \leq 2  \right\} 
    \qquad
    B = \left\{ z \in \mathbb{C}; \left|z - \frac{1}{2} \right| \geq \frac{1}{2} \right\}    
    \]
    Poglejmo najprej množico $B$. Predstavlja vsa števila, ki so za vsaj $\frac{1}{2}$ oddaljena od točke $\left( \frac{1}{2}, 0 \right)$. Torej predstavlja zunanjost kroga z radijem $\frac{1}{2}$ in središčem v tej točki (in tudi to krožnico).
    
    Pri množici $A$, pa je verjetno koristno, da najprej še malo poračunamo, da bomo lažje videli kaj predstavlja. Pišemo $z = x + iy$.
    \begin{align*}
        \frac{1}{x + iy} + \frac{1}{x - iy} &\leq 2\\
        \frac{(x - iy) + (x + iy)}{(x + iy)(x - iy)} &\leq 2\\
        \frac{2x}{x^2 - y^2i^2} = \frac{2x}{x^2 + y^2} &\leq 2\\
    \end{align*}
    Imenovalec ulomka je vedno pozitiven, zato lahko z njim množimo.
    \begin{align*}
        2x &\leq 2x^2 + 2y^2\\
        0 &\leq x^2 - x + y^2 = (x - \frac{1}{2})^2 - \frac{1^2}{2^2} + y^2\\
        \left( \frac{1}{2} \right)^2 &\leq (x - \frac{1}{2})^2 + y^2
    \end{align*}
    Spet dobimo neenačbo, ki predstavlja kompleksna števila, ki ležijo v zunanjosti iste krožnice (ali na tej krožnici).


    \item Zapiši $x^5 - 1$ kot produkt realnih polinomov stopnje največ 2.\\
    Najprej naivno poskusimo preprosto razbiti polinom s hornerjevim algoritmom. Z ``metodo ostrega pogleda'' najdemo eno ničlo, to je $x = 1$.
    \[
    x^5 - 1 = (x - 1)(x^4 + x^3 + x^2 + x + 1)
    \]
    Sedaj moramo samo faktorizirati še preostali del polinoma. Kako pa bi našli še kakšno ničlo? (\emph{malo prelistam zvezek\ldots})
    Zakaj le bi bila taka naloga med kompleksnimi števili? Vrnimo se na začetek naloge (včasih je koristno napisati tudi kakšno napako, iz napak se navsezadnje učimo). Lahko samo obrnemo enačbo:
    \[
    x^5 = 1
    \]
    Sedaj pa poiščemo 5-te korene števila 1, za katere ta enakost velja.
    \begin{align*}
    e^{i 5 \varphi} &= e^{i \cdot 0}\\
    5\varphi &= 0 + 2k\pi,\ k = 0, 1, 2, 3, 4\\
    \varphi &= \frac{2k\pi}{5}
    \end{align*}
    Sedaj lahko zapišemo vse rešitve. Vemo, da kompleksne rešitve v polinomih z realnimi koeficienti nastopajo v urejenih parih, tako da jih zapišemo tako.
    \begin{align*}
    x_0 &= 1 &\\
    x_1 &= \cos \frac{2\pi}{5} + i \sin \frac{2\pi}{5} &
    x_2 &= \cos \frac{2\pi}{5} - i \sin \frac{2\pi}{5} \\
    x_3 &= \cos \frac{4\pi}{5} + i \sin \frac{4\pi}{5} &
    x_4 &= \cos \frac{4\pi}{5} - i \sin \frac{4\pi}{5}
    \end{align*}
    Da dobimo polinome z realnimi koeficienti, moramo zmnožiti po dve ničli. To ni zares težko, ampak najprej raje naredimo to na preprostem primeru, da dobimo formulo, potem pa vstavimo zgornje ničle.
    \begin{multline*}
        (x - x_i) (x - \overline{x_i}) = (x - a - ib) (x - a + ib) =\\
        = x^2 - ax + bxi - ax + a^2 - bxi + abi + b^2 =\\
        = x^2 - 2ax + a^2 + b^2 
    \end{multline*}
    Ker sta naši komponenti ničel $\cos \zeta$ in $\sin \zeta$ za nek kot $\zeta$, vemo da velja, da je vsota njunih kvadratov enaka 1.
    \begin{align*}
    (x - x_1) (x - x_2) &= x^2 - 2 x \cos \frac{2\pi}{5} + 1\\
    (x- x_3)(x - x_4) &= x^2 - 2 x \cos \frac{4\pi}{5} + 1
    \end{align*}
    Torej je naš končni polinom:
    \[
    x^5 - 1 = \left(x - 1\right)\left( x^2 - 2x \cos \frac{2\pi}{5} + 1\right) \left(x^2 - 2x \cos \frac{4\pi}{5} + 1\right)
    \]
    
    
    \item Z uporabo Moivrove formule izrazi $\cos 4\varphi$ in $\sin 4\varphi$ samo z $\sin \varphi$ in $\cos \varphi$.\\
    Načeloma bi lahko nalogo rešili zgolj tako, da bi dvakrat uporabili formule za dvojne kote, vendar to ne bi bilo zares poučno. Bolj koristno je, če si najprej pogledamo kako bi lahko to rešili za splošen večkratnik kota $\varphi$ (recimo mu $n$).
    Recimo, da bi šli računat tole kompleksno število:
    \begin{align*}
        (\cos \varphi + i \sin \varphi)^n &= \l &= \ldots  + i(\ldots)\\
        \intertext{Dobili bi nekaj ogabnega, ampak to znamo izračunat s pomočjo binomske formule. Lahko pa bi uporabili Moivrovo formulo:}
        (\cos \varphi + i \sin \varphi)^n &= \cos n\varphi + i \sin n \varphi
    \end{align*}
    Vidimo, da je realni del ogabne stvari zgoraj natanko enak $\cos n \varphi$ in imaginarni del enak $\sin n \varphi$. Sedaj moramo le še ugotoviti, kateri členi izraza $(\cos \varphi + i \sin \varphi)^n$ pri razvoju v binomsko vrsto so realni in kateri imaginarni. Tega ni težko ugotoviti in je prepuščeno bralcu v razmislek (namig: stopnja $i$ v izrazu se spreminja, dobimo lahko tudi pozitivne in negativne člene).
    
    Za stopnjo 4 pa bomo brez večjih težav to poračunali tudi na roke, oziroma naj bi znali tudi formulo:
    \[
    (\cos \varphi + i \sin \varphi)^4 = \cos^4 \varphi + 4i \cos^3 \varphi \sin \varphi - 6 \cos^2 \varphi \sin^2 \varphi - 4i \cos \varphi \sin^3 \varphi + \sin^4 \varphi
    \]
    Torej samo še poberemo vsake člene posebej in dobimo rešitev:
    \begin{align*}
    \cos 4 \varphi &= \cos^4 \varphi - 6 \cos^2 \varphi \sin^2 \varphi + \sin^4 \varphi\\
    \sin 4 \varphi &= 4\cos^3 \varphi \sin \varphi - 4 \cos \varphi \sin^3 \varphi
    \end{align*}

    \item Naj bo $a = \frac{1}{2} + i \frac{\sqrt{3}}{2}$. Poišči vse rešitve enačbe $z^7 + z^5a + z^3a^2 + za^3$.
    
    To nalogo se mi zdi da smo že rešili na proseminarju. Samo na hitro bom napisal postopek: na daleč vidimo, da je ena od rešitev $z = 0$. Ko se znebimo te rešitve, potem lahko naredimo substitucijo, $w = z^2$. Dobimo enačbo, ki je na nek način simetrična, moramo samo pametno izpostaviti člene da jo rešimo.
    \[
    w^3 + w^2a + wa^2 + a^3 = w(w^2 + a^2) + a(w^2 + a^2) = (w^2 + a^2)(w + a) 
    \]
    Obravnavamo dva pogoja, pri katerih je enačba lahko enaka 0. Rešimo za $z$.
     
     
    \item Poišči vsa kompleksna števila, ki izpolnjujejo enačbi:
    \begin{align*}
    |z|^2 + (2 + i)z  + (2 - i)\overline{z} + 4 &= 0 \\
    (1-  i)z + (1 + i)\overline{z} + 4 &= 0    
    \end{align*} 
    Torej iščemo presek rešitev teh dveh enačb. No pa rešimo ti dve enačbi (pišemo $z = x + iy$):
    \begin{align*}
    &|z|^2 + (2 + i)z  + (2 - i)\overline{z} + 4 = \\
    &= x^2 + y^2 + (2 + i)(x + iy) + (2 - i)(x - iy) + 4 =\\
    &= x^2 + y^2 + 2x + 2iy + xi - y + 2x - 2iy - xi - y + 4 =\\
    &= x^2 + y^2 + 4x - 2y + 4 = \\
    &= (x + 2)^2 - 4 + (y - 1)^2 - 1 + 4 = 0\\
    &\qquad (x + 2)^2 + (y - 1)^2 = 1
    \end{align*} 
    Prva enačba torej predstavlja krožnico s središčem $S(-2, 1)$ in radijem 1.
    \begin{align*}
    &(1-  i)(x + iy) + (1 + i)(x - iy) + 4 =\\
    &= x + iy - ix  + y + x -iy + ix + y + 4 =\\
    &= 2x + 2y + 4 = 0\\
    &\qquad y = -x -2
    \end{align*} 
    Druga enačba pa predstavlja premico. Načeloma bi lahko šli računat presečišča premice in krožnice, vendar v tem primeru to ni potrebno, saj so presečišča lepa in jih lahko razberemo iz skice (ki je žal ne bo v tem dokumentu).
    \[
    z_1 = -2 + 0i\qquad z_2 = -3 + 1i
    \]

    \item Poišči vsa kompleksna števila $z$, ki ustrezajo pogojema:
    \[
    z^2 + \overline{z} + 1 = 0, \quad \text{Im}(z^2) \geq 1
    \]
    Spet iščemo presek dveh množic. Najprej malo poračunajmo, da bomo videli, kaj ti množici predstavljata.
    \begin{align*}
    z^2 + \overline{z} + 1 &= 0\\
    x^2 - y^2 + 2xyi + x - iy + 1 &= 0
    \end{align*} 
    \begin{itemize}
        \item $x^2 - y^2 + x + 1 = 0$: Če enačbo malo preuredimo, vidimo, da to predstavlja hiperbolo.
        \[
        (x + \frac{1}{2})^2 - y^2 = -\frac{3}{4} 
        \]
        \item $2xy - y = 0$: Če izpostavimo $y$ dobimo 2 možnosti:
        
        \begin{itemize}
            \item $y = 0$
            \item $2x  - 1 = 0 \implies x = \frac{1}{2}$
        \end{itemize}
    \end{itemize}
    Rezultat prve enačbe so točke, za katere velja enačba za realni in imaginarni del. Poglejmo še drugo enačbo:
    \[
    \text{Im}(x^2 - y^2 + 2xyi) = 2xy \geq 1
    \]
    Poiščimo vse točke, ki jim to ustreza:
    \begin{itemize}
        \item $y = 0$: V tem primeru neenačba $2xy \geq 1$ ne velja. Ni rešitev.
        \item $x = \frac{1}{2}$:
        \[
        1^2 - y^2 = -\frac{3}{4} \implies y = \frac{\sqrt{7}}{2}
        \]
        Poglejmo še, če velja za 2. enačbo:
        \[
        2 \cdot \frac{1}{2} \cdot \frac{\sqrt{7}}{2} = \frac{\sqrt{7}}{2} \geq 1
        \]
    \end{itemize}
    Torej je rešitev naloge $z = \frac{1}{2} + i \frac{\sqrt{7}}{2} $

\item
Naj bodo $z$, $w$ in $t$ kompleksna števila. Pokaži, da velja 
	\[
    	z\cdot Im(\overline{w}t) + w\cdot Im(\overline{t}z) + t\cdot Im(\overline{z}w) = 0
    \]
    Uporabimo predlagano formulo $Im z = \frac{z-\overline{z}}{2i}$ in zapišemo:
    \begin{align*}
    	z\cdot\frac{\overline{w}t - \overline{\overline{w}t}}{2i} + w\cdot\frac{\overline{t}z - 	\overline{\overline{t}z}}{2i} + t\cdot\frac{\overline{z}w - \overline{\overline{z}w}}{2i} &=  \\
        \intertext{Znebimo se dvojnih konjugiranih vrednosti in poračunamo:}
        =z\cdot\frac{\overline{w}t - \overline{t}w}{2i} + w\cdot\frac{\overline{t}z - \overline{z}t}{2i} + t\cdot\frac{\overline{z}w - \overline{w}z}{2i} &=  \\
        =\frac{tz\overline{w} - zw\overline{t} + zw\overline{t} - tw\overline{z} + tw\overline{z} - tz\overline{w}}{2i} &=  \\
        \intertext{Kot vidimo, so členi v števci paroma nasprotno predznačeni:}
        =\frac{tz\overline{w} - tz\overline{w} + zw\overline{t} - zw\overline{t} + tw\overline{z} - tw\overline{z}}{2i} &=  \\
        =\frac{0+0+0}{2i} &= 0
    \end{align*}
\end{enumerate}
\end{document}
