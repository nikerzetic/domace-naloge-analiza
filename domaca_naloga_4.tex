\documentclass[12pt,a4paper,slovene]{article}
\usepackage [slovene]{babel}
\usepackage[utf8]{inputenc}

\usepackage{enumerate}
\usepackage{amsmath}
\usepackage{mathtools}
\usepackage{amssymb}

\newcommand{\NN}{\mathbb{N}}
\newcommand{\RR}{\mathbb{R}}
\newcommand{\ZZ}{\mathbb{Z}}

\begin{document}
\begin{enumerate}[(1)]
	\item Poenostavi naslednje izraze
	\begin{enumerate}[(a)]
		\item $(3+4i)\overline{(1-3i)}$
		\begin{multline*}
			(3+4i)\overline{(1-3i)} = (3+4i)(1+3i) = 3 + 9i + 4i - 12 = -9 + 13i
		\end{multline*}
		
		\item $\dfrac{7 - 3i}{1 + i}$
		
		\begin{multline*}
			\dfrac{7-3i}{1+i} = \dfrac{(7-3i)(1-i)}{(1+i)(1-i)} = \dfrac{7 - 7i - 3i - 3}{2} = \dfrac{4 - 10i}{2} = 2 - 5i
		\end{multline*}
		
		\item $(1 + i \sqrt{3})^{10}$
		
		To ena"cbo najla"zje re"simo tako, da jo pretvorimo v eulerjev zapis.
		\begin{align*}
		z & = 1 + i \sqrt{3}\\
		|z| &= \sqrt{1 + 3} = 2 \\
		\tan \varphi &= \dfrac{\sqrt{3}}{1} \Rightarrow \varphi = \dfrac{\pi}{3}\\
		z &= 2e^{i\frac{\pi}{3}}
		\end{align*}
		Nato pora"cunamo njeno deseto potenco:
		\begin{equation*}
		z^{10} = 2^{10}e^{10i\frac{\pi}{3}} = 1024e^{i\frac{4\pi}{3}}
		\end{equation*}
		Nato lahko to pretvorimo nazaj v kartezi"cno obliko:
		\begin{equation*}
		z^{10} = 1024\left(\cos\dfrac{4\pi}{3} + i\sin\dfrac{4\pi}{3}\right) =  1024\left(-\dfrac{1}{2} - i\dfrac{\sqrt{3}}{2}\right) = -512\left(1 + i\sqrt{3}\right)
		\end{equation*}
	\end{enumerate}

	\item Zapi"si vse vrednosti izraza $\sqrt[7]{(-\sqrt{3} - i)^5}$ in jih nari"si.
	
	To nalogo ponovno najla"zje re"simo tako, da pretvorimo v eulerjev zapis.
	\begin{align*}
	w &= -\sqrt{3} - i\\
	z^7 &= (-\sqrt{3} - i)^5 = w^5
	\end{align*}
	Re"sujemo za $z$.
	
	Najprej pora"cunamo $w^5$:
	\begin{align*}
	w &= -\sqrt{3} - i\\
	|w| &= \sqrt{3 + 1} = 2\\
	\tan \varphi &= \dfrac{-1}{-\sqrt{3}} \Rightarrow \varphi = \dfrac{\pi}{6}\\
    \end{align*}
    \textbf{Pozor!} Tule se skriva pogosta napaka. Nismo izračunali pravega kota. Če bi narisali skico, bi takoj opazili, da pri številu $w = -\sqrt{3} - i$ kot zagotovo ne more biti $\frac{\pi}{6}$! Pravi kot je večji za $\pi$, torej je $\varphi = \frac{7 \pi}{6}$. Sedaj lahko zapišemo $w$ v eulerjevi obliki.
    \begin{align*}
    w &= 2e^{i\frac{7\pi}{6}}\\
    w^5 &= 2^5 e^{i\frac{35\pi}{6}} = 32 e^{i\frac{35\pi}{6}}\\
    \intertext{najti želimo 7-mi koren tega števila}
    z &= |z|e^{i\varphi}\\
	z^7 &= w^5\\
	|z|^7 e^{i7\varphi} &= 32 e^{i(\frac{35\pi}{6} + 2k\pi)}\qquad k \in \ZZ
	\end{align*}
	Sedaj primerjamo absolutne vrednosti in kote posebej. Za absolutne vrednosti dobimo:
	\begin{equation*}
	|z|^7 = 32 \Rightarrow |z| = \sqrt[7]{32}
	\end{equation*}
	Za kote pa dobimo:
	\begin{align*}
	7\varphi &= \frac{35\pi}{6} + 2k\pi\\
	\varphi &= \frac{5\pi}{6} + \dfrac{2k\pi}{7},\ k \in \{0, 1, 2, \ldots , 6\}
	\end{align*}
	$k$ je celo število od $0$ do $6$, ker ko je $k = 7$ pri"stejemo kotu $2\pi$ in iz trigonometrije vemo, da smo nazaj na za"cetku.
	
	$z$ lahko sedaj zapi"semo kot:
	\begin{equation*}
	z = \sqrt[7]{32} e^{i\left(\frac{5\pi}{6} + \frac{2k\pi}{7}\right)} = \sqrt[7]{32}\left(\cos \left(\dfrac{5\pi}{6} + \dfrac{2k\pi}{7}\right) +i\sin \left(\dfrac{5\pi}{6} + \dfrac{2k\pi}{7} \right)\right)
	\end{equation*}
	 	
	\item Skiciraj naslendje podmno"zice kompleksnih "stevil.
	
    \begin{enumerate}[(a)]
        \item $\{z \in \mathbb{C}; 2 (\text{Re } z)^2 + \text{Im } < 1 \}$\\
        Zapišimo $z$ v kartezični obliki, $z = x + iy$. Vidimo, da so v množici tista kompleksna števila, ki ustrezajo neenačbi:
        \[
        2x^2 + y < 1
        \]
        Enačbo še malo preuredimo, in že znamo narisati to množico. Tu je žal ne bom narisal, ker trenutno nimam preveč časa.
        \[
        y < -2x^2 + 1
        \]
        Rezultat so vse točke, ki ležijo pod parabolo $y = -2x^2 + 1$.
        
        \item $\{ z \in \mathbb{C}; \text{Re }z^2 + 4 \text{Im }z = 0 \}$
        Naredimo enako kot pri prejšnji nalogi, samo da imamo celo nekaj dela z računanjem.
        \begin{align*}
            \text{Re}(x + iy)^2 + 4y &= 0\\
            \text{Re}(x^2 - y^2 + 2xyi) + 4x = x^2 - y^2 + 4y &= 0\\
            x^2 - (y - 2)^2 + 4 &= 0\\
            (x + 2)^2 - y^2 &= -4\\
            \frac{(x+2)^2}{4} - \frac{y^2}{4} &= -1
        \end{align*}
        Po tem, ko dopolnimo popolni kvadrat je očitno, da gledamo v enačbo hiperbole. Risanje je prepuščeno bralcu.
    \end{enumerate}
	
	\item Naj bo $k \neq 0$ realno "stevilo. Za katere $k$ je $1 + ik$ bli"ze izhodi"s"cu kot $1-\frac{i}{k}$?
	
	Razdaljo od izhodi"s"ca nam predstavlja absolutna vrednost "stevila, torej lahko zapi"semo:
	\begin{equation*}
	|1+ik| < \left|1-\dfrac{i}{k}\right|
	\end{equation*}
	Ker so absolutne vrednosti ve"cje ali enake 0, lahko neena"cbo kvadriramo in razre"simo absolutne vrednost:
	\begin{equation*}
	1 + k^2 < 1 + \dfrac{1}{k^2}
	\end{equation*}
	Sedaj lahko ena"cbo nekoliko preuredimo in dobimo:
	\begin{align*}
	k^2 &< \dfrac{1}{k^2}\\
	k^4 &< 1
	\end{align*}
	Ker je $k^2$ pozitivno "stevilo, lahko ena"cbo korenimo in dobimo:
	\begin{equation*}
	k^2 < 1
	\end{equation*}
	Sledi dolg premislek o tem, kaj je na"s rezultat. "Ce si skiciramo parabolo $k^2 = 1$, dobimo dve ni"cli: $-1, 1$, parabola pa je med tema dvema ni"clama manj"sa od 1. Torej je $k \in (-1, 1)$, ali druga"ce povedano: $|k| < 1$.
    
    Seveda tudi pri tej nalogi obstaja lahka očitna geometrijska rešitev. Narišemo si premico, na kateri ležijo vsa kompleksna števila, $1 + ik$, ko $k$ preteče vsa realna števila (razen 0). To je premica $x = 1$. Hitro opazimo da, manjši kot je $|k|$, manjša je oddaljenost od izhodišča. Sedaj nas zanima samo še za katere $k$ velja:
    \[
    |k| < \frac{1}{|k|}
    \]
    Očitno to velja samo za števila $|k|$, ki so $\leq 1$.

    \item Rešeno na faksu pri proseminarju. Nekako nima smisla da bi še enkrat pisal rešitev.
    
    \item Izračunaj množice kompleksnih števil $z$, ki zadoščajo (ne)enačbam. Množice rešitev nariši.
    
    \begin{enumerate}[(a)]
        \item $z^3 = -2 + 2i$
        Obe strani enačbe zapišemo z eulerjevo formulo in primerjamo absolutno vrednost ter kot.
        \begin{align*}
            |-2 + 2i| &= \sqrt{(-2)^2 + 2^2} = 2\sqrt{2}\\
            \tan \theta &= \left(\frac{y}{x}\right) = -\frac{2}{2} = -1 \implies \theta = \frac{3\pi}{4}\\
            \left(|z| e^{i \varphi}\right) &= |z|^3 e^{i 3\varphi} = 2\sqrt{2} e^{i\frac{3\pi}{4}}\\
            |z|^3 &= 2\sqrt{2} = \sqrt{8} \implies |z| = \sqrt[6]{8} = \sqrt{2}\\
            3\varphi &= \frac{3\pi}{4} + 2k\pi, k \in \{0, 1, 2\}\\
            \varphi &= \frac{\pi}{4} + \frac{2 k \pi}{3}              
        \end{align*}
        Sedaj lahko še napišemo števila z, ki ustrezajo enačbi, vendar v se ni treba preveč mučit, napišimo samo tako, kot je v rešitvah:
        \[
        z = \sqrt{2}(\cos(\frac{\pi}{4} + \frac{2k\pi}{3}) + i \sin(\frac{\pi}{4} + \frac{2k\pi}{3})),\ k = 0, 1, 2
        \]
        
        \item $z^8 + z^4 - 12 = 0$\\
        Poskusimo podobno, vendar preden se vržemo v nalogo lahko naredimo preprosto substitucijo, da nam bo lažje. Naj bo $w = z^4$. Tako moramo rešiti samo preprosto kvadratno enačbo.
        \begin{align*}
            w^2 + w - 12 &= 0\\
            (w + 4) (w - 3) &= 0 \implies w = -4 \lor w = 3 
        \end{align*}
        Zdaj samo poračunamo za oba primera.
        \begin{enumerate}
            \item $w=-4$
            \begin{align*}
                z^4 &= -4\\
                |z| &= \sqrt[4]{4} = \sqrt{2}\\
                \varphi &= \frac{\pi}{4} + \frac{2\pi}{4} = \frac{pi}{4} + \frac{\pi}{2}\\
                z &= \sqrt{2} (\cos(\frac{\pi}{4} + \frac{\pi}{2}) + i \sin (\frac{\pi}{4} + \frac{\pi}{2})), \ k = 0, 1, 2, 3
            \end{align*}
            \emph{Opomba:} Če bi slučajno pomislil, potem bi opazil, da so tole precej lepi koti, prav tako razdalja od izhodišča, vse nam je že od nekod znano. Lahko pišemo tudi:
            \[
            z = \pm 1 \pm i
            \]
            \item $w = 3$\\
            Identično kot za prvi primer, samo da se meni ne da toliko pisat.
            \begin{align*}
                z &= \sqrt{3} (\ldots)\\
                \varphi &= \frac{\pi}{4} + \frac{\pi}{2}\\
                z &= \sqrt{3} (\cos\varphi + i\sin \varphi),\ k = 1, 2, 3, 4\\
            \end{align*}
        \end{enumerate}
        
        \item $\sqrt{|z|^2 - 2} + \text{Im}\;z > \text{Re}\;z$\\
        Poskusimo zapisati $z$ v kartezični obliki, torej $z = x + yi$:
        \begin{align*}
            \sqrt{x^2 + y^2 - 2} + y > x\\
            \sqrt{x^2 + y^2 - 2} > x - y\\
        \end{align*}
        Poglejmo, kdaj ima ta enačba sploh smisel. Pogledati moramo, kdaj je vrednost znotraj korena negativna, in kakšna je desna stran.
        %
        \begin{enumerate}
            \item $x - y < 0$: Desna stran enačbe je negativna, leva pa bo vedno nenegativna, torej bo neenačaj veljal na celotnem definicijskem območju, torej ko velja $x^2 + y^2 - 2 \geq 0$.
            \[
            (x < y) \land (x^2 + y^2 \geq 2)
            \]
            \item $x - y \geq 0$: Enačbo lahko kvadriramo.
            \begin{align*}
                x^2 + y^2 - 2 &> (x - y)^2 = x^2 - 2xy +  y^2\\
                -2 &> -2xy\\
                xy &> 1\\
                (x \geq y) &\land (xy > 1)
            \end{align*}
        \end{enumerate}
        Torej imamo celotno rešitev:
        \[
        \left\{ z = x+ iy; (x \geq y) \land (xy > 1) \lor (x < y) \land (x^2 + y^2 \geq 2) \right\}
        \]
        
    \end{enumerate}

\end{enumerate}
\end{document}
