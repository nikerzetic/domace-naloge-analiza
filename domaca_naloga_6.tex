% Used latex template from https://github.com/DzinVision/latex-templates/blob/master/article.tex

% Set font size
\documentclass[a4paper, 12pt]{article}
% Font and language settings
\usepackage[slovene]{babel}
\usepackage[utf8]{inputenc}
\usepackage[T1]{fontenc}
\usepackage{lmodern}
% Links
\usepackage{hyperref}
\usepackage{url}
% Images
\usepackage{graphicx}
% Better controls over figure envirnoments
\usepackage{float}
% Various math symobs
\usepackage{amssymb}
\usepackage{amsmath}
\usepackage{amsthm}
\usepackage{mathtools}
% Multiple text columns
\usepackage{multicol}
% Miscellaneous symobls
\usepackage{wasysym}
% Custom enumeration symols
\usepackage{enumerate}
% Able to indent text with adjustwidth environment
\usepackage{changepage}
% More dashing option. Most important is \dashuline{} used for short proofs
\usepackage[normalem]{ulem}

% Comment this out if you want normal paragraph indents without empty lines between them.
% This options are set primarly for texts with a lot of shorter paragraphs, like various notes.
\setlength{\parindent}{0px}
\setlength{\parskip}{10px}

\usepackage{titlesec}
\titleformat{\section}{\bfseries}{}{0pt}{(\thesection)\quad}

% Some macros for common number sets
\newcommand{\NN}{\ensuremath{\mathbb{N}}}
\newcommand{\ZZ}{\ensuremath{\mathbb{Z}}}
\newcommand{\QQ}{\ensuremath{\mathbb{Q}}}
\newcommand{\RR}{\ensuremath{\mathbb{R}}}
\newcommand{\CC}{\ensuremath{\mathbb{C}}}


\begin{document}
    \begin{center}
        \Large\textbf{6. domača naloga}
    \end{center}

    \section{Naj bo $a_n = \frac{2}{3n + 7}$. Napiši nekaj členov zaporedja $a_n$. Ugotovi, ali je zaporedje navzgor omejeno, navzdol omejeno, naraščajoče, padajoče in, ali je konvergentno. Če je konvergento, izračunaj limito.}
    
    Zapisati nekaj členov ne bi smelo biti pretežko.
    \[
    a_1 = \frac{2}{3 + 7} = \frac{2}{10}, \quad a_2 = \frac{2}{6 + 7} = \frac{2}{13}, a_3 = \frac{2}{9 + 7} = \frac{2}{16} \ldots
    \]
    Iz tega ugibamo, da je zaporedje padajoče. To dokažemo tako, da pokažemo, da je vsak naslednji člen manjši od prejšnjega.
    \begin{align*}
        a_{n+1} &\stackrel{?}\leq a_n \\
        \frac{2}{3n + 3 + 7} &\stackrel{?}\leq \frac{2}{3n + 7} \\
        \intertext{Ni težav z množenjem, ker imamo sama pozitivna naravna števila}
        2(3n + 7) &\stackrel{?}\leq 2(3n + 10) \\
        3n + 7 &\stackrel{?}\leq 3n + 10 \\
        7 &\leq 10
    \end{align*}
    Torej je zaporedje padajoče (in ni naraščajoče). Je navzgor omejeno z $a_1 = \frac{1}{5}$. Z metodo ostrega pogleda opazimo, da je tudi navzdol omejeno, saj bo katerikoli člen zaporedja $a_n$ zagotovo pozitiven. Spodnja meja je 0.
    
    Ker je zaporedje navzdol omejeno in padajoče, ima limito. Ker je to eden lažjih primerov, jo lahko uganemo: $\lim_{n \to \infty} a_n = 0$. Pa po definiciji dokažimo, da je 0 limita, saj je tudi to koristno znati.
    
    Izberemo si poljuben $\varepsilon > 0$ za katerega pokažemo, da za vse $n \in \NN$ večje od nekega števila velja:
    \begin{align*}
    |a_n - a| < \varepsilon \\
    \left| \frac{2}{3n + 7} - 0\right| = \frac{2}{|3n + 7|} < \varepsilon 
    \intertext{Vemo, da je $3n + 7$ pozitivno število ne glede na $n$.}
    \frac{2}{3n + 7} < \varepsilon
    \end{align*}
    To velja zaradi arhimedske lastnosti. Sledi $\lim_{n \to \infty} a_n = 0$.
    
    \section{Po definiciji dokaži:}
    \[
    \lim_{n \to \infty} \frac{3n + 1} {2n-1} = \frac{3}{2}
    \]
    Ker smo se v srednji šoli od daleč dotaknili limit, vemo da ta enakost velja. Poskusimo jo dokazati na enak način kot smo to naredili pri prejšnji nalogi.
    
    Izberemo poljuben $\varepsilon > 0$
    \begin{align*}
        \left|  \frac{3n + 1} {2n-1} - \frac{3}{2} \right| &< \varepsilon \\
        \left|  \frac{3n + 1- \frac{3}{2}(2n-1)} {2n-1}\right| = 
        \left|  \frac{3n - 3n + 1 + \frac{3}{2}} {2n-1} \right| = 
        \left| - \frac{5} {4n-2} \right| &< \varepsilon \\
        \intertext{Upoštevamo absolutno vrednost. Vemo da sta števec in imenovalec vedno pozitivna.}
        \frac{5} {4n-2} &< \varepsilon \\
        \intertext{To velja zaradi arhimedske lastnosti, venar lahko še malo poračunamo, da bo lepše vidno.}
        \frac{5} {\varepsilon} &< 4n - 2 \\
        \frac{5} {4\varepsilon} + \frac{1}{2} &< n
    \end{align*}
    Enakost torej velja za vsa naravna števila večja od leve strani enačbe.
    

\end{document}
