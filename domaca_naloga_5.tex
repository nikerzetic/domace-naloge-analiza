% Used latex template from https://github.com/DzinVision/latex-templates/blob/master/article.tex

% Set font size
\documentclass[a4paper, 12pt]{article}
% Font and language settings
\usepackage[slovene]{babel}
\usepackage[utf8]{inputenc}
\usepackage[T1]{fontenc}
\usepackage{lmodern}
% Links
\usepackage{hyperref}
\usepackage{url}
% Images
\usepackage{graphicx}
% Better controls over figure envirnoments
\usepackage{float}
% Various math symobs
\usepackage{amssymb}
\usepackage{amsmath}
\usepackage{amsthm}
\usepackage{mathtools}
% Multiple text columns
\usepackage{multicol}
% Miscellaneous symobls
\usepackage{wasysym}
% Custom enumeration symols
\usepackage{enumerate}
% Able to indent text with adjustwidth environment
\usepackage{changepage}
% More dashing option. Most important is \dashuline{} used for short proofs
\usepackage[normalem]{ulem}

% Comment this out if you want normal paragraph indents without empty lines between them.
% This options are set primarly for texts with a lot of shorter paragraphs, like various notes.
\setlength{\parindent}{0px}
\setlength{\parskip}{10px}

\usepackage{titlesec}
\titleformat{\section}{\bfseries}{}{0pt}{(\thesection)\quad}

% Some macros for common number sets
\newcommand{\NN}{\ensuremath{\mathbb{N}}}
\newcommand{\ZZ}{\ensuremath{\mathbb{Z}}}
\newcommand{\QQ}{\ensuremath{\mathbb{Q}}}
\newcommand{\RR}{\ensuremath{\mathbb{R}}}
\newcommand{\CC}{\ensuremath{\mathbb{C}}}


\begin{document}
    \begin{center}
        \Large\textbf{5. domača naloga}
    \end{center}

    \section{Dokaži, da imata množici $\RR$ in $\RR \times \{0, 1\}$ isto moč.}
    Želimo pokazati, da imata 2 množici isto moč. To lahko naredimo na več načinov. Lahko konstruiramo bijekcijo iz ene množice v drugo, s čimer bi dokazali, da imata množici enako moč. Vendar bijekcije ni vedno lahko najti. Zato lahko uporabimo kratek izrek z dolgim imenom:
    \[
    \text{Cantor-Bernstein-Schroederjev izrek:}\quad
    |A| \leq |B| \land |B| \leq |A| \implies |A| = |B|
    \]     
    Pri čemer oznaka $|A|$ predstavlja moč množice $A$. Z drugimi besedami to pomeni, da lahko pokažemo, da sta množici $A$ in $B$ enako močni tako, da poiščemo:
    \begin{itemize}
        \item Injektivno preslikavo $A \to B$ in
        \item Injektivno preslikavo $B \to A$
    \end{itemize}
    Iz tega sledi, da mora med množicama $A$ in $B$ obstajati tudi neka bijekcija, čeprav ni nujno, da smo jo našli.
    
    \begin{itemize}
        \item $\RR \to \RR \times \{0, 1\} \quad :\quad x \mapsto (x, 0)$
        
        To je primer injektivne preslikave. Ni pomembno, katero smo našli, zanima nas le, če se vsaka 2 elementa iz množice $\RR$ preslikata v 2 različna elementa množice $\RR \times \{0, 1\}$. To v tem primeru očitno velja.
        \item $\RR \times \{0, 1\} \to \RR$:
        
        Kako pa bi sedaj naredili injektivno preslikavo? Pri vajah smo pokazali, da velja:
        \[
        |(0, 1)| = |\RR|
        \]
        Torej je dovolj, da pokažemo, da obstaja takšna injektivna preslikava:
        \[
        (0, 1) \times \{0, 1\} \to \RR
        \]
        Takšne preslikave pa ni težko konstruirati. Primer:
        \[
        (a, b) \mapsto a + b
        \]
        \textbf{Opomba:} Zakaj velja $|(0, 1)| = |\RR|$? Ni težko konstruirati bijekcije, ki bi preslikala elemente nekega intervala v vsa realna števila. To lahko naredimo npr. z eno vejo funkcije $\tan$, ki jo ustrezno premaknemo in raztegnemo/skrčimo v $x$ smeri.
        \[
        f(x) = \tan \pi (x - \frac{1}{2})\ ,\qquad
        f^{-1}(x) = \frac{1}{\pi} \arctan x + \frac{1}{2}
        \]
        \textbf{Še ena opomba:} Ni potrebno, da $\RR$ zamenjamo z manjšim intervalom. V tem primeru lahko samo uporabimo neko funkcijo, ki ``splošči'' vsa realna števila na nek končen interval. Primer:
        \[
        (a, b) \mapsto \arctan a + 10 b
        \]
    \end{itemize}
    
    \section{Dokaži, da imata množici $\RR$ in $(\RR \times \{0\}) \cup (\ZZ \times [0, 1])$ isto moč.}
    
    Enako kot pri prejšnji nalogi, konstruirajmo 2 injekciji na teh dveh množicah:
    \begin{itemize}
        \item $\RR \to (\RR \times \{0\}) \cup (\ZZ \times [0, 1]) \quad : \quad x \mapsto (x, 0)$
        \item $(\RR \times \{0\}) \cup (\ZZ \times [0, 1]) \to \RR$:
        
        Spet imamo več težav s tem, da bi pokazali obratno. Problema se lahko rešimo tudi tako, da napišemo 2 posamezna predpisa, za vsak del množice posebej:
        \[
        (a, b) \mapsto \left\{ \begin{array}{ll}
            \arctan a - 10 & \text{če } (a, b) \in \RR \times \{0\} \\
            a + \frac{b}{4} & \text{če } (a, b) \in \ZZ \times [0, 1] \text{ in } a \geq 0 \\
            -a + \frac{b}{4} + \frac{1}{2} & \text{če } (a, b) \in \ZZ \times [0, 1] \text{ in } a < 0 \\
            \end{array}
            \right.
        \]
        To je samo en primer, kako bi lahko konstruirali takšno injektivno preslikavo. Prepričan sem, da se da nalogo rešiti na kakšen veliko bolj eleganten način. Poglejmo zakaj je zgornja preslikava injektivna:
        \begin{itemize}
            \item Elementi $\RR \times \{0\}$ se bodo injektivno preslikali na interval \\$[-10-\frac{\pi}{2},-10 + \frac{\pi}{2}]$
            \item Elementi  $\ZZ \times [0, 1]$ se bodo preslikali v neskončno mnogo zaprtih intervalov, vendar samo v nenegativna realna števila.
            \[
            \left[0, \frac{1}{4}\right] \cup \left[1, 1+\frac{1}{4}\right] \cup \left[1+\frac{1}{2}, 1+\frac{3}{4}\right] \cup \left[2, 2+\frac{1}{4}\right] \cup \ldots
            \]
        \end{itemize}
        Torej je ta preslikava injektivna.
    \end{itemize}

    \section{Dokaži, da imata množici $[0, 1]$ in $\{(x, y)\in\RR \mid x^2 + y^2 =1 \}$ isto moč.}
    \begin{itemize}
        \item $[0, 1] \to \{(x, y)\in\RR \mid x^2 + y^2 =1 \}$
        
        V to stran injektivne preslikave ni tako težko narediti, vsako točko na intervalu preslikamo v točko na krožnici nad intervalom.
        \[
        x \mapsto (x, \sqrt{1 - x^2})
        \]
        \item $\{(x, y)\in\RR \mid x^2 + y^2 =1 \} \to [0, 1]$
        
        Če poskusimo z inverzno funkcijo funkcije $x \mapsto (x, \sqrt{1 - x^2})$, vidimo, da dobro deluje za vse točke v prvem kvadrantu, težave pa imamo z vsemi ostalimi točkami. Lahko pa preprosto ``stisnemo'' točke iz prvega kvadranta samo na manjši del intervala $[0, 1]$, ter tako naredimo ``prostor'' za ostale točke.
        \[
        (x, y) \mapsto \left\{ \begin{array}{ll}
            \frac{x}{8} & x \geq 0 \land y \geq 0 \\
            \frac{x}{8} + \frac{1}{4} & x < 0 \land y \geq 0 \\
            \frac{x}{8} + \frac{1}{2} & x < 0 \land y < 0 \\
            \frac{x}{8} + \frac{3}{4} & x \geq 0 \land y < 0
        \end{array} \right.
        \]
        Spet pa je to samo ena izmed mnogo možnosti, kako konstruirati injektivno funkcijo.
    \end{itemize}

    \section{Dokaži, da imata množici $[0, 1]^2$ in $[0, 1] \times [0, 1)$ isto moč.}
    \label{naloga4}
    \begin{itemize}
        \item $[0, 1] \times [0, 1) \to [0, 1]^2$
        
        Dokažimo najprej v to stran, ker je malo lažje. Injektivna preslikava se ponuja sama od sebe: 
        \[
        (a, b) \mapsto (a, b)\qquad \text{Ker: } [0, 1] \times [0, 1) \subseteq [0, 1]^2
        \]        
        \item $[0, 1]^2 \to [0, 1] \times [0, 1)$
        
        V to stran pa lahko naredimo tako, da drugi element v paru, malo ``stisnemo'' v manjši interval:
        \[
        (a, b) \mapsto \left(a, \frac{b}{2} \right)
        \]
    \end{itemize}

    \section{Dokaži, da imata množici $[0, 1]^2$ in $[0, 1)^2$ isto moč.}
    S pomočjo rešene naloge \ref{naloga4} je trivialno rešiti to nalogo, zato je reševanje za vajo prepuščeno bralcu.
    
    \section{Ali je katera izmed naslednjih množic števna?}
    Kdorkoli že si, te pred branjem sledeče naloge opozarjam na visoko verjetnost napake. Zatorej ti priporočam uporabo kritičnega mišljenja.
    \begin{itemize}
        \item $X = \{(x_1, x_2, x_3, \ldots ) \mid x_i \in \ZZ \}$
        
        Podobno nalogo smo delali na vajah, zato posumimo, da je množica števno neskončna. Če je to res, je treba dokazati, da je moč te množice enaka moči realnih števil. To lahko naredimo tako, da opišemo neko konstrukcijo, s katero bi naredili injektivni preslikavi.
        
        \begin{itemize}
            \item $\RR \to X$
            
            Naj bo $r \in \RR$. Iščemo preslikavo, ki bo $r$ preslikala v neko zaporedje celih števil. Število $r$ lahko zapišemo v decimalnem zapisu.
            \[
            r = r_0, r_1 r_2 r_3 \ldots r_k r_{k+1} \ldots
            \]
            $r_0$ je gotovo celo število. Števke $r_1, r_2  \ldots$ so števila med 0 in 9. torej so prav tako cela števila. Lepo bi bilo opozoriti še na primer, ko je $r$ oblike $r_0,999999\ldots$ kjer lahko nastopijo težave. Načeloma je preslikava v vsakem primeru injektivna, vendar ni zares več preslikava, saj enako število, zapisano na 2 različna načina preslika v 2 različni sliki. Primer:
            \[
            1 = 0,\overline{9} \qquad (1, 0, 0, \ldots) \neq (0, 9, 9, \ldots)
            \]
            
            \item $X \to \RR$
            
            Preslikavo v drugo smer lahko definiramo podobno, če imamo v mislih prejšnji primer. Imamo zaporedje celih števil, vendar bomo v tem primeru naredili samo preslikavo iz zaporedja naravnih števil. To je enako, saj ima množica $\NN$ enako moč kot $\ZZ$. 
            \[
            (x_1, x_2, x_3, x_4, \ldots), \quad x_i \in \NN
            \] 
            Vsako naravno število pa lahko v decimalnem zapisu zapišemo s končno mnogo števkami.
            \[
            (x_{1_n} \ldots x_{1_2} x_{1_1} x_{1_0},\quad
            x_{2_k} \ldots x_{2_2} x_{2_1} x_{2_0}, \quad\ldots)
            \]
            Zdaj pa lahko vse števke zapišemo ``povrsti'', ter tako dobimo neko realno število:
            \[
            r = 0, x_{1_n} \ldots x_{1_2} x_{1_1} x_{1_0}
                    x_{2_k} \ldots x_{2_2} x_{2_1} x_{2_0} \ldots
            \]
        \end{itemize}
        
        \item $Y_n = \{ (x_1, x_2, x_3, \ldots ) \in X \mid x_i = 0 \text{ za } i > n \}$
        
        Ta množica ima eno pomembno razliko od prejšnje. Vsa zaporedja imajo vse člene od nekje naprej enake 0. To nas spomni na racionalna števila, kar pa je števno neskončna množica.
        
        Pravzaprav lahko na to množico potem gledamo drugače:
        \[
        Y_n = \ZZ ^ n \times \{0\} ^{|\ZZ|}
        \]
        Za moč takšne množice pa že vemo, da je števno neskončna ($\ZZ^2 = \QQ$ je števna, torej je tudi $\QQ^2 = \ZZ^4$ števno neskončna\ldots).
        \item $Y = \bigcup_{n=1}^\infty Y_n$
        
        Sedaj pa gledamo množico ki je pravzaprav: 
        \begin{align*}
        &\ZZ \times \{0\} \times \{0\} \times \{0\} \times \ldots \\
        \cup\ &\ZZ \times \ZZ \times \{0\} \times \{0\}  \times \ldots \\
        \cup\ &\ZZ \times \ZZ \times \ZZ \times \{0 \} \times \ldots \\
        \cup\ &\ldots
        \end{align*}
        Sedaj pa želimo še ugotoviti, ali je ta množica števno neskončna ali ne. Lahko poskusimo zapisati vse elemente. Začnimo s prvo vrstico. Z $a_{1_i}$ označimo člene zaporedja prvi vrstici.
        \[
        a_{1_1}, a_{1_2}, a_{1_3}, a_{1_4}, \ldots \qquad a_{1_i} \in \ZZ \times \{0 \} \times \ldots
        \]
        Enako naredimo 2. vrstico, potem še s tretjo itd.
        \begin{align*}
        &a_{1_1}, a_{1_2}, a_{1_3}, a_{1_4}, \ldots \\
        &a_{2_1}, a_{2_2}, a_{2_3}, a_{2_4}, \ldots \\
        &a_{3_1}, a_{3_2}, a_{3_3}, a_{3_4}, \ldots \\
        &\vdots
        \end{align*}
        Spet uporabimo isti trik kot pri dokazovanju, da so racionalna števila števno neskončna množica. Začnemo v kotu zgoraj levo, potem pa po diagonalah razporedimo vsa števila v vrsto:
        \[
        a_{1_1}, a_{1_2}, a_{2_1}, a_{1_3}, a_{2_2}, a_{3_1}, a_{1_4}, a_{2_3} \ldots
        \]

    \end{itemize}
   

\end{document}
