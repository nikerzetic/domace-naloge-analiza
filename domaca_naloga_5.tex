% Used latex template from https://github.com/DzinVision/latex-templates/blob/master/article.tex

% Set font size
\documentclass[a4paper, 12pt]{article}
% Font and language settings
\usepackage[slovene]{babel}
\usepackage[utf8]{inputenc}
\usepackage[T1]{fontenc}
\usepackage{lmodern}
% Links
\usepackage{hyperref}
\usepackage{url}
% Images
\usepackage{graphicx}
% Better controls over figure envirnoments
\usepackage{float}
% Various math symobs
\usepackage{amssymb}
\usepackage{amsmath}
\usepackage{amsthm}
\usepackage{mathtools}
% Multiple text columns
\usepackage{multicol}
% Miscellaneous symobls
\usepackage{wasysym}
% Custom enumeration symols
\usepackage{enumerate}
% Able to indent text with adjustwidth environment
\usepackage{changepage}
% More dashing option. Most important is \dashuline{} used for short proofs
\usepackage[normalem]{ulem}

% Comment this out if you want normal paragraph indents without empty lines between them.
% This options are set primarly for texts with a lot of shorter paragraphs, like various notes.
\setlength{\parindent}{0px}
\setlength{\parskip}{10px}

\usepackage{titlesec}
\titleformat{\section}{\bfseries}{}{0pt}{(\thesection)\quad}

% Some macros for common number sets
\newcommand{\NN}{\ensuremath{\mathbb{N}}}
\newcommand{\ZZ}{\ensuremath{\mathbb{Z}}}
\newcommand{\QQ}{\ensuremath{\mathbb{Q}}}
\newcommand{\RR}{\ensuremath{\mathbb{R}}}
\newcommand{\CC}{\ensuremath{\mathbb{C}}}


\begin{document}
    \begin{center}
        \Large\textbf{5. domača naloga}
    \end{center}

    \section{Dokaži, da imata množici $\RR$ in $\RR \times \{0, 1\}$ isto moč.}
    Želimo pokazati, da imata 2 množici isto moč. To lahko naredimo na več načinov. Lahko konstruiramo bijekcijo iz ene množice v drugo, s čimer bi dokazali, da imata množici enako moč. Vendar bijekcije ni vedno lahko najti. Zato lahko uporabimo kratek izrek z dolgim imenom:
    \[
    \text{Cantor-Bernstein-Schroederjev izrek:}\quad
    |A| \leq |B| \land |B| \leq |A| \implies |A| = |B|
    \]     
    Pri čemer oznaka $|A|$ predstavlja moč množice $A$. Z drugimi besedami to pomeni, da lahko pokažemo, da sta množici $A$ in $B$ enako močni tako, da poiščemo:
    \begin{itemize}
        \item Injektivno preslikavo $A \to B$ in
        \item Injektivno preslikavo $B \to A$
    \end{itemize}
    Iz tega sledi, da mora med množicama $A$ in $B$ obstajati tudi neka bijekcija, čeprav ni nujno, da smo jo našli.
    
    \begin{itemize}
        \item $\RR \to \RR \times \{0, 1\} \quad :\quad x \mapsto (x, 0)$
        
        To je primer injektivne preslikave. Ni pomembno, katero smo našli, zanima nas le, če se vsaka 2 elementa iz množice $\RR$ preslikata v 2 različna elementa množice $\RR \times \{0, 1\}$. To v tem primeru očitno velja.
        \item $\RR \times \{0, 1\} \to \RR$:
        
        Kako pa bi sedaj naredili injektivno preslikavo? Pri vajah smo pokazali, da velja:
        \[
        |(0, 1)| = |\RR|
        \]
        Torej je dovolj, da pokažemo, da obstaja takšna injektivna preslikava:
        \[
        (0, 1) \times \{0, 1\} \to \RR
        \]
        Takšne preslikave pa ni težko konstruirati. Primer:
        \[
        (a, b) \mapsto a + b
        \]
        \textbf{Opomba:} Zakaj velja $|(0, 1)| = |\RR|$? Ni težko konstruirati bijekcije, ki bi preslikala elemente nekega intervala v vsa realna števila. To lahko naredimo npr. z eno vejo funkcije $\tan$, ki jo ustrezno premaknemo in raztegnemo/skrčimo v $x$ smeri.
        \[
        f(x) = \tan \pi (x - \frac{1}{2})\ ,\qquad
        f^{-1}(x) = \frac{1}{\pi} \arctan x + \frac{1}{2}
        \]
        \textbf{Še ena opomba:} Ni potrebno, da $\RR$ zamenjamo z manjšim intervalom. V tem primeru lahko samo uporabimo neko funkcijo, ki ``splošči'' vsa realna števila na nek končen interval. Primer:
        \[
        (a, b) \mapsto \arctan a + 10 b
        \]
    \end{itemize}
    
    \section{Dokaži, da imata množici $\RR$ in $(\RR \times \{0\}) \cup (\ZZ \times [0, 1])$ isto moč.}
    
    Enako kot pri prejšnji nalogi, konstruirajmo 2 injekciji na teh dveh množicah:
    \begin{itemize}
        \item $\RR \to (\RR \times \{0\}) \cup (\ZZ \times [0, 1]) \quad : \quad x \mapsto (x, 0)$
        \item $(\RR \times \{0\}) \cup (\ZZ \times [0, 1]) \to \RR$:
        
        Spet imamo več težav s tem, da bi pokazali obratno. Problema se lahko rešimo tudi tako, da napišemo 2 posamezna predpisa, za vsak del množice posebej:
        \[
        (a, b) \mapsto \left\{ \begin{array}{ll}
            \arctan a - 10 & \text{če } (a, b) \in \RR \times \{0\} \\
            a + \frac{b}{4} & \text{če } (a, b) \in \ZZ \times [0, 1] \text{ in } a \geq 0 \\
            -a + \frac{b}{4} + \frac{1}{2} & \text{če } (a, b) \in \ZZ \times [0, 1] \text{ in } a < 0 \\
            \end{array}
            \right.
        \]
        To je samo en primer, kako bi lahko konstruirali takšno injektivno preslikavo. Prepričan sem, da se da nalogo rešiti na kakšen veliko bolj eleganten način. Poglejmo zakaj je zgornja preslikava injektivna:
        \begin{itemize}
            \item Elementi $\RR \times \{0\}$ se bodo injektivno preslikali na interval \\$[-10-\frac{\pi}{2},-10 + \frac{\pi}{2}]$
            \item Elementi  $\ZZ \times [0, 1]$ se bodo preslikali v neskončno mnogo zaprtih intervalov, vendar samo v nenegativna realna števila.
            \[
            \left[0, \frac{1}{4}\right] \cup \left[1, 1+\frac{1}{4}\right] \cup \left[1+\frac{1}{2}, 1+\frac{3}{4}\right] \cup \left[2, 2+\frac{1}{4}\right] \cup \ldots
            \]
        \end{itemize}
        Torej je ta preslikava injektivna.
    \end{itemize}

    \section{Dokaži, da imata množici $[0, 1]$ in $\{(x, y)\in\RR \mid x^2 + y^2 =1 \}$ enako moč.}
    \begin{itemize}
        \item $[0, 1] \to \{(x, y)\in\RR \mid x^2 + y^2 =1 \}$
        
        V to stran injektivne preslikave ni tako težko narediti, vsako točko na intervalu preslikamo točko na krožnici nad intervalom.
        \[
        x \mapsto (x, \sqrt{1 - x^2})
        \]
        \item $\{(x, y)\in\RR \mid x^2 + y^2 =1 \} \to [0, 1]$
        
        Če poskusimo z inverzno funkcijo funkcije $x \mapsto (x, \sqrt{1 - x^2})$, vidimo, da dobro deluje za vse točke v prvem kvadrantu, težave pa imamo z vsemi ostalimi točkami. Lahko pa preprosto ``stisnemo'' točke iz prvega kvadranta samo na manjši del intervala $[0, 1]$, ter tako naredimo ``prostor'' za ostale točke.
        \[
        (x, y) \mapsto \left\{ \begin{array}{ll}
            \frac{x}{8} & x \geq 0 \land y \geq 0 \\
            \frac{x}{8} + \frac{1}{4} & x < 0 \land y \geq 0 \\
            \frac{x}{8} + \frac{1}{2} & x < 0 \land y < 0 \\
            \frac{x}{8} + \frac{3}{4} & x \geq 0 \land y < 0
        \end{array} \right.
        \]
        Spet pa je to samo ena izmed mnogo možnosti, kako konstruirati injektivno funkcijo.
    \end{itemize}


\end{document}
